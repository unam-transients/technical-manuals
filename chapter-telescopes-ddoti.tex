\chapter{Telescopes}
\label{chapter:telescopes}

\section{Description}

The {\projectname} telescopes are six Celestron 11-inch Rowe-Ackermann Schmidt Astrograph (RASA) telescopes each equipped with a Starlight Instruments HandyMotor and FocusBoss II focuser and a custom-made detector adapter. The telescopes are mounted on the mount using a custom-made adapter.

The six telescopes are associated with six detector channels. Channels C0, C2, and C4 are mounted on one side of the mount and channels C1, C3, and C5 are mounted on the other. 

The original two UNAM telescopes were installed in June 2017. The remaining four UMD telescopes were installed in February 2019. We also have a seventh telescope on site. This was dropped during installation of the UMD telescope and the corrector plate was destroyed. It is now used for spares. A replacement UNAM telescope was installed in March 2019.

Table~\ref{table:telescopes} gives the serial and inventory numbers of the telescopes. The telescopes to have two serial numbers, one on a silver-colored label and one on a gold-colored label.

\begin{table}
\caption{Telescopes}
\label{table:telescopes}
\begin{center}
\begin{tabular}{cccc}
\hline
\hline
Channel&Serial Numbers&Owner&Inventory Number\\
\hline
C0&985394/110780&UMD&\phantom{00}243595\\
C1&985164/110661&UMD&\phantom{00}243588\\
C2&980740/105806&UNAM&02480886\\
C3&\phantom{000000}/108014&UNAM&02333800\\
C4&985163/110200&UMD&\phantom{00}243587\\
C5&984230/110102&UMD&\phantom{00}243589\\
\hline
Spare&&UNAM&\\
\hline
\end{tabular}
\end{center}
\end{table}

The telescope optical properties are shown in Table~\ref{table:telescope-optical-properties}. The telescopes use the default filter window.

\begin{table}
\caption{Telescope Optical Properties}
\label{table:telescope-optical-properties}
\begin{center}
\begin{tabular}{ll}
\hline
\hline
Property&Value\\
\hline
Diameter&279 mm (11 inches)\\
Central Obscuration Diameter (Telescope)&114 mm\\
Central Obscuration Diameter (Detector Adapter)&132 mm\\
Central Obscuration Ratio (Telescope)&0.41\\
Central Obscuration Ratio (Detector Adapter)&0.47\\
Focal Length&620 mm\\
Focal Ratio&$f/2.2$\\
Optimized Field Diameter&43.3 mm (4.0 degrees)\\
\hline
\end{tabular}
\end{center}
\end{table}

We have very little information on the transmission of the telescope. According to the RASA white paper, the primary mirror has a “Celestron StarBright XLT” coating and the corrector plate, corrector lenses, and filter have “broadband AR” coatings, but we have no quantitative information on the actual performance. Nevertheless, the telescope image quality was optimized for 400--700~nm and it makes sense that the coatings were also optimized for this range.

\section{Mount Adapter}

\section{Optical Alignment}

The telescopes have three optical groups: the corrector plate, the spherical mirror, and the four-element corrector lens group. The corrector plate and spherical mirror have no alignment mechanism. The corrector lenses have a three-point push-pull mechanism to allow collimation and the reduction of coma.

We aligned the corrector lenses in March 2019. We mounted a Sony $\alpha$6000 camera using an E-to-T mount adapter and the Celestron-supplied T-mount RASA adapter. We observed out of focus stars at the center of the field using magnified live-view and adjusted the mechanism until the secondary obscuration was centered.

\section{Hartmann Tests}

\section{Detector Adapter}

\section{Focusers}

\section*{Bibliography}

\begin{flushleft}
\begin{itemize}
\item “\href{bibliography/celestron-rasa-11-manual.pdf}{Celestron RASA Instruction Manual}”, 2014.
\item “\href{bibliography/celestron-rasa-11-white-paper.pdf}{Celestron RASA White Paper}”, 2016.
\item “\href{bibliography/celestron-rasa-11-camera-adapter.pdf}{Celestron RASA Recommendations for Camera Adapter}”, 2016.
\end{itemize}
\end{flushleft}
